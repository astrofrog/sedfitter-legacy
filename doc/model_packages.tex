\documentclass[11pt]{article}
\usepackage[left=2.0cm,top=3.0cm,right=2.0cm,bottom=3.0cm,nohead,a4paper]{geometry}
\renewcommand{\familydefault}{\sfdefault}

\setlength{\parindent}{0pt}

\RequirePackage[calcwidth]{titlesec}
\titleformat{\subsection}[block]{\sffamily\bfseries}{\large}{12pt}{\\\large}[{\titlerule[0.5pt]}]


\begin{document}

\title{SED fitter model packages}
\author{Thomas Robitaille}
\date{9 June 2008}
\maketitle

This document describes the format required of a model `package' in order to be used by the fitter. The format adopted for all files is the \texttt{FITS} standard. All FITS files presented here should have \texttt{BITPIX=-32} and \texttt{EXTEND=T} in the primary header. The model `package' should have a name starting with \texttt{model\_} and containing the following:

\begin{itemize}

\item \texttt{seds/} - a directory containing all the model SEDs as \texttt{fits.gz} files, in the format described below. In order to avoid huge numbers of files in a single directory, sub-directories can be created with a section of the model names. The size of this section can later be specified in the models config file. In the \texttt{models\_r06} package, the \texttt{seds} directory contains sub-directories whose filename is the five first characters of the model name. These sub-directories then contain the appropriate models. For example, the \texttt{30034} directory contains all models starting with \texttt{30034}.

\item \texttt{convolved/} - a directory containing all the convolved models, one file per filter. The name of the files should be \texttt{filter.fits} where \texttt{filter} should be replaced with the filter code (e.g. \texttt{2J}, \texttt{2H}, etc.).

\item \texttt{parameters.fits.gz} - the parameters for the models, in the same order as the convolved fluxes

\end{itemize}

\section{SED files}

\subsection{HDU 0}

The primary header must contain following keywords:

\begin{verbatim}
VERSION  = 1   (integer) (the current version, in case we make big changes in future)
MODEL    = value (character) (the model name)
IMAGE    = T/F (logical) (whether images are available in HDU 0)
WAVLGHTS = T/F (logical) (whether wavelengths are available in HDU 1)
APERTURS = T/F (logical) (whether apertures are available in HDU 2)
SEDS     = T/F (logical) (whether SEDs are available in HDU 3)
\end{verbatim}

if \texttt{WAVLGHTS = T}, then the header must contain

\begin{verbatim}
NWAV = value (integer) which is the number of wavelengths
\end{verbatim}

if \texttt{APERTURS = T}, then the header must contain

\begin{verbatim}
NAP = value (integer) which is the number of apertures
\end{verbatim}

if \texttt{IMAGE = T}, then this \textbf{HDU} should contain a data cube which is the
model image as a function of wavelength. This will be specified in more detail in future.

\subsection{HDU 1 - \texttt{extname=WAVELENGTHS}}

if \texttt{WAVLGTHS = T}, this \textbf{HDU} should contain a 2-column binary table with \texttt{NWAV}
values in each column. The first column should have the title \texttt{WAVELENGTH}, format \texttt{1E}, and unit \texttt{MICRONS}. The second column should have the title \texttt{FREQUENCY}, format \texttt{1E}, and unit \texttt{HZ}. Optionally, a third column can be included with the title \texttt{STELLAR\_FLUX}, format \texttt{1E}, and units specified appropriately. This column can be used to specify the model stellar photosphere used to compute YSO models for example. Only \texttt{mJy} and \texttt{ergs/cm\^{}2/s} are supported as units at this time. If specified, the stellar photosphere should have the correct scaling relative to the SEDs in \texttt{HDU 3}.


\subsection{HDU 2 - \texttt{extname=APERTURES}}

if \texttt{APERTURS = T}, this \textbf{HDU} should contain a 1-column binary table with \texttt{NAP}
values. The column should have the title \texttt{APERTURE}, format \texttt{1E}, and units \texttt{AU}. These are the apertures for which the SEDs in \textbf{HDU 3} are tabulated.

\subsection{HDU 3 - \texttt{extname=SEDS}}

if \texttt{SED = T}, this \textbf{HDU} should contain a binary table with at least one column, and \textbf{HDU 1} and  \textbf{HDU 2}  should also contain data. Each column should consist of \texttt{NAP} rows of real vectors with dimension \texttt{NWAV}. Thus, each cell contains an SED. The format of each column should be \texttt{nE}, where \texttt{n=NWAV}. The title and units of each column should be specified. The columns can contains SEDs such as the total flux, the stellar flux, the disk flux, etc. or related uncertainties. The following column titles are examples of ones that can be used:

\begin{verbatim}
TOTAL_FLUX
TOTAL_FLUX_ERR
STELLAR_FLUX
STELLAR_FLUX_ERR
DISK_FLUX
DISK_FLUX_ERR
ENVELOPE_FLUX
ENVELOPE_FLUX_ERR
DIRECT_FLUX
DIRECT_FLUX_ERR
SCATTERED_FLUX
SCATTERED_FLUX_ERR
THERMAL_FLUX
THERMAL_FLUX_ERR
etc.
\end{verbatim}

The order of the columns is not important as there are \texttt{FITS} routines to search for a specific column.\\

\textbf{Note : } The SED fitter requires a column \texttt{TOTAL\_FLUX} to be present, and will return an error otherwise. Only \texttt{mJy} and \texttt{ergs/cm\^{}2/s} are supported as units at this time.

\section{Convolved fluxes file}

\subsection{HDU 0}

The primary header must contain following keywords:

\begin{verbatim}
FILTWAV  = value (real) (the characteristic wavelength of the filter)
NMODELS  = value (integer) (the number of models)
NAP      = value (integer) (the number of apertures)
\end{verbatim}

\subsection{HDU 1  - \texttt{extname=CONVOLVED FLUXES}}

This \textbf{HDU} should contain a 5-column binary table. The column titles should be:

\begin{verbatim}
MODEL_NAME
TOTAL_FLUX
TOTALF_FLUX_ERR
RADIUS_SIGMA_50
RADIUS_CUMUL_99
\end{verbatim}

The first column should have format \texttt{30A} and should contain the name of each model. No units are required. The second and third columns should have format \texttt{nE} where \texttt{n=NAP}, with each cell containing a vector with the fluxes in the different apertures. The fourth and fifth column should have format \texttt{1E} and contain the outermost radius at which the surface brightness falls to 50\% of the maximum surface brightness, and the radius inside which 99\% of the flux is contained respectively. These two columns should have units \texttt{AU}.

\subsection{HDU 2 - \texttt{extname=APERTURES}}

This \textbf{HDU} should contain a 1-column binary table with \texttt{NAP}
values. The column should have the title \texttt{APERTURE}, format \texttt{1E}, and units \texttt{AU}. These are the apertures for which the fluxes in \textbf{HDU 1} are tabulated.

\section{Model parameters}

\subsection{HDU 0}

The primary header must contain following keywords:

\begin{verbatim}
NMODELS  = value (integer) (the number of models)
\end{verbatim}

\subsection{HDU 1  - \texttt{extname=CONVOLVED FLUXES}}

This \textbf{HDU} should contain a binary table with the model parameters. Any number of columns can be included, in any order. Only parameters with format \texttt{1E} will be usable by the programs to plot parameters, but text parameters with format \texttt{nA} can also be included (e.g. dust model filenames, etc.). One column is compulsory, with title \texttt{MODEL\_NAME} and format \texttt{30A}. It should contain the same names as the convolved fluxes file, and in the same order.

\section{Fitter output file}

\subsection{HDU 0}

The primary header must contain following keywords:

\begin{verbatim}
MODELDIR = path  (character) (the models directory used)
EXLAW    = path  (character) (the extinction law used)
OUT_FORM = value (character) (the output format - N/C/D/E/F etc.)
OUT_NUMB = value (real) (the number associated with the output format)
\end{verbatim}

\subsection{HDU 1}

The header must contain following keywords:

\begin{verbatim}
NWAV  = value   (integer) (number of wavelengths)
\end{verbatim}

and for each wavelength \texttt{i}:

\begin{verbatim}
FILTi = value (character) (the filter code)
WAVi  = value (real) (the characteristic wavelength)
APi   = value (real) (the aperture in arcseconds)
\end{verbatim}

This \texttt{HDU} should contain a binary table with 11 columns:

\begin{verbatim}
title                unit          format      description
--------------------------------------------------------------------------------
SOURCE_NAME                         30A        source name
X                    deg             1D        x position
Y                    deg             1D        y position
SOURCE_ID                            1J        id used in HDU 2
FIRST_ROW                            1J        first row in HDU 2
NUMBER_FITS                          1J        number of fits listed
VALID                                nJ        valid flag for data
FLUX                 mJy             nE        flux
FLUX_ERROR           mJy             nE        flux error
LOG10FLUX            mJy             nE        log10 flux (bias corrected)
LOG10FLUX_ERROR      mJy             nE        log10 flux error (bias corrected)
\end{verbatim}

with one row per source, where n=NWAV (the four flux related columns should contain vectors of size \texttt{NWAV}).

\subsection{HDU 2}

The following header keyword should be specified:

\begin{verbatim}
MODELFLX = T/F (logical) (whether the model fluxes are specified in this HDU
\end{verbatim}

This \textbf{HDU} should contain at least 6 columns:

\begin{verbatim}
title                unit        format      description
--------------------------------------------------------------------------------
SOURCE_ID                          1J        id listed in HDU 1
FIT_ID                             1J        rank of the fit
MODEL_ID                           1J        model ID (model # in convolved flux file)
MODEL_NAME                        30A        name of the model
CHI2                               1E        total chi-squared
AV                   mag           1E        magnitudes of visual extinction
LOGD                 kpc           1E        log(d) distance required for fit
\end{verbatim}

optionally, a 7th column can be specified (if \texttt{MODELFLX=T}), with name \texttt{MODEL\_LOG10FLUX}, containing the scaled and extincted model fluxes that fit the data. The format of the column should be \texttt{nE} where \texttt{n=NWAV}, and the units should be \texttt{mJy}.


\end{document}
